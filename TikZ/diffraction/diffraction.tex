\documentclass[dvipdfmx, 
				border = {5pt, 5pt, 5pt, 5pt}]{standalone}	% borderで余白をあける

\usepackage{tikz}	% ティクス
\usepackage{amsmath,amsthm,amssymb}
\usepackage{bm}
\usepackage{ascmac}
\usepackage{tikz-3dplot}

\usetikzlibrary{intersections, math, patterns, calc, angles, quotes}
\usetikzlibrary{decorations.pathmorphing}

% intersections .. 交点の座標を求めるため
% math .. 計算を行う
% patterns .. 領域をパターンで塗りつぶすため


% ~~~~>
\makeatletter
\newcommand*\snakerightarrow[2][]{%
\mathrel{%
\settowidth\@tempdima{\footnotesize#1}%
\settowidth\@tempdimb{\footnotesize#2}%
\ifdim\@tempdimb>\@tempdima\@tempdima=\@tempdimb\fi
\advance\@tempdima15pt
\tikz[font=\footnotesize,baseline=-2pt]
\draw[decorate,decoration={snake, pre length=3pt, post length=3pt, segment length=6pt, amplitude=2pt},->] 
(0,0) -- node[above] {#2} node[below] {#1} (\@tempdima,0);%
}%
}
\makeatother


\newcommand{\angmark}[4]{
 \coordinate(Aang)at($(#1)-(#2)$);
 \coordinate(Bang)at($(#3)-(#2)$);
 \draw let \p1=(Aang), \p2=(Bang)
       in ($(#2)!0.5cm!(#1)$) arc
          [start angle={atan2(\x1,\y1)}, end angle={atan2(\x2,\y2)}, radius=0.5cm];
}

\newcommand{\contrangmark}[4]{
 \coordinate(Aang)at($(#1)-(#2)$);
 \coordinate(Bang)at($(#3)-(#2)$);
 \draw
  let \p1=(Aang), \p2=(Bang),
      \n1={atan2(\x1,\y1)}, \n2={atan2(\x2,\y2)+360}
  in ($(#2)!0.5cm!(#1)$)
     arc [start angle={\n1}, end angle={\n2}, radius=0.5cm];
 \draw[opacity=0]
  let \p1=(Aang), \p2=(Bang),
      \n1={atan2(\x1,\y1)}, \n2={atan2(\x2,\y2)+360},
      \n3={0.5*\n1+0.5*\n2}
  in ($(#2)!.8cm!(#1)$)
     arc [start angle={\n1}, end angle={\n3}, radius=.8cm]
     coordinate(Cang);
 \draw(Cang)node{#4};
}



\begin{document}

\tdplotsetmaincoords{0}{0}
\begin{tikzpicture}[tdplot_main_coords, domain = -1.5:4, samples = 100, thick]	% sampleじゃなくてsamples


  % 斜線を引く
   %\fill [top color = white, bottom color = white, middle color = purple]
	% plot[sample = 100, domain = 0:4] (\x, {\x / 2 + 2})
	%(4, 4)--(4, 0)--(1, 0)--(0, 1) --(0, 2) -- (4, 4);

  % 斜線を引く
  % \fill [pattern = north east lines, very thin]		% 薄くしたいんだが
  %	% plot[sample = 100, domain = 0:4] (\x, {\x / 2 + 2})
  %	(4, 4)--(4, 0)--(1, 0)--(0, 1) --(0, 2) -- (4, 4);

  % 軸の描画
  \draw [thick, -stealth] (0, 0)--(0, 3) node[above] {\large $x^\prime$};			% x'軸
  \draw [thick, -stealth] (0, 0)--(-1.8, -1.8) node[below left] {\large $y^\prime$};		% y'軸

  \draw [thick, -stealth] (9, 0)--(9, 4.5) node[above] {\large $x$};					% x軸
  \draw [thick, -stealth] (9, 0)--(6.5, -2.5) node[below left] {\large $y$};				% y軸

  \draw [thick] (-1.5, 0)--(-1, 0);
  \draw [thick] (0, 0)--(7.5, 0);
  \draw [thick, -stealth] (9, 0)--(13, 0) node[right] {\large $z$};					% z軸


  % 平面波
  \draw (-3.1, 0.25) node {$\snakerightarrow{\large 平面波}$};


  % 板
  \draw [thick] (-1, 1)--(-1, -3)--(1, -1)--(1, 3)--(-1, 1);
  \draw [thick] (7.5, 1.5)--(7.5, -4.5)--(10.5, -1.5)--(10.5, 4.5)--(7.5, 1.5);

  % ベクトル
  \draw [very thick, -stealth] (0, 0)--(9.5, 1.8);
  \draw [very thick, -stealth] (9, 0)--(9.5, 1.8) node [above right, fill=white] {\large $(x, y)$};
  \draw [very thick, -stealth] (0, 0)--(-0.7, 1) node [above left, fill=white] {\large $(x^\prime, y^\prime)$};

  \draw (4.5, 1) node[above]  {$\bm{R}$};			% R
  \draw (9.3, 0.8) node[right] {$\bm{\rho}$};			% ρ
  \draw (-0.31, 0.37) node[left] {$\bm{\rho}^\prime$};	% ρ'


  \draw [dashed] (9.5, 1.8)--(0.5, 1.8);
  \draw [dashed] (0, 0)--(0.5,1.8);

  % 座標と角
  \coordinate (prime) at (-0.7, 1);
  \coordinate (rho) at (0.5, 1.8);
  \coordinate (O) at (0, 0);

  \draw  pic["\large $\phi^\prime$",draw, -stealth, thick, angle eccentricity=1.45, angle radius=0.7cm] 
		{angle=rho--O--prime};		% phi

  \coordinate (R) at (9.5, 1.8);
  \coordinate (z) at (1, 0);l
  \draw pic["\large $\theta$", draw, -stealth, thick, angle eccentricity=1.2, angle radius=2cm]
		{angle=z--O--R};


  \draw (0, 2.1) node[left]{\large $S^\prime$};		% S'
  \draw (9, 3.1) node[left]{\large $S$};		% S

  % 白く塗る
  \fill [white]		% 矢印Rを隠す
	(7.52, 1)--(7.52, 1.45)--(8.9, 1.78)--(8.9, 1) --(7.52, 1);
  

  \draw (0, -3.2) node[below] {\large 開口関数 \ $f(x^\prime, y^\prime)$};
  \draw (9, -3.2) node[fill=white, below] {\large 振幅分布 \ $u(x, y)$};
  
  
  % 黒点
%    \foreach \i in {0, 1, 2, 3, 4, 5, 6, 7}
%      \foreach \j in {0, 1, 2, 3}
%        \fill (\i, \j) circle [radius=2.5pt];

  % 白丸
%    \foreach \i in {0, 1, 2, 3, 4, 5, 6, 7}
%      \foreach \j in {-2, -1}
%        \draw[fill=white] (\i, \j) circle (2.5pt);
  
      
  % \draw [thin, dashed] (1, -0.8)--(1, 4.5)
  %	node [left, fill = white, inner sep = 0pt] at(0.8, 4) {$x_1 = \frac{k}{\alpha}$};
  

  % 関数の描画
  % \draw [domain = -1.2:2] plot(\x, {-\x + 1}) 
  % 	node [right, fill = white, inner sep = 0pt]  at(2.1, -0.9) {$x_1 + x_2 = 1$};

  
  % 領域D
  % \draw (2.1, 1.5) node [fill = white] {$D$};

  % 座標
  % \node [below left] at(1, 0) {1};
  % \node [left] at(0, 1) {1}; 
  % \node [above left] at(0, 2) {2};		% left above ではだめ

\end{tikzpicture}

\end{document}
