% ch0-preface.tex
\documentclass[../main/main]{subfiles}

\begin{document}
\small

\chapter*{はじめに}
\addcontentsline{toc}{chapter}{はじめに}		% 「はじめに」を目次に表示

本稿は物理学で頻繁に現れるHermite多項式、Laguerre(陪)多項式、Legendre(陪)多項式、
(変形, 球)Bessel関数についての基礎公式とその証明を与えたものです。
これらは「特殊関数」と呼ばれ、実用の上では少なくとも以下のすべての項目に習熟する必要があります。\vspace{3pt}
\begin{itemize}
  \item 母関数
  \item Rodriguesの公式
  \item 一般項
  \item 漸化式
  \item 微分方程式
  \item 直交性
\end{itemize}\vspace{3pt}
ところが、現行の教科書, 参考書ではこれらの項目および各公式の間の相互関係について
断片的な記述がなされているものや
証明が与えられていないものがほとんどであり、
なかなか公式間の関係や全体像をつかむことができず、
雑多な公式の数々に学習者が尻込みしてしまう現状にあります。

そこでこれら特殊関数に関する必要最低限度の公式だけを選び、
それぞれの公式がどの公式からどのようにして導かれるものであるかについてを
冗長にならない程度にできるだけ省略することなく記すことにしてみました。
公式を暗記して使うことは本質的ではないのであまり求められることはありませんが、
各項目間の論理的な道筋を実際に計算してたどることは
それぞれの特殊関数に慣れ親しむことにたいへん役に立ちます。
また各導出過程では大学初年度水準の数学をふんだんに用いるため、
その復習をする上でも大いに価値があると思います。

\section*{本稿の構成}
第1章から第4章までは特殊関数の性質とその証明をひたすら書いています。
本稿の大部分は、各関数ごとに先に使用頻度の高い公式をすべて示しておき、
その後にそれらの公式を導くような形式を取っています。
そのため、本稿は教科書というよりは証明付き公式集と呼んだほうがよりその特徴を表していると思います。
このような形式を採用した理由は、
各公式ごとに証明を付す形式にするとレファレンスとして利用しにくくなると感じたからです。
ただし、はじめの公式集にまとめて入れておくことが難しいと思われた
いくつかの項目(Legendre多項式の多重極展開など)は別の節を新たに設けてそこで紹介することにしました。

第5章では第4章までに学んだ特殊関数がどのように応用されているのかについての具体例を、
問題形式でいくつか紹介しています。
実際の問題をいくつか解いてみることで特殊関数にさらに親しむことができると思います。




\section*{想定する読者}
本稿においては次のような人を読者として想定しています。 \vspace{3pt}
\begin{itemize}
  \item 特殊関数を勉強したことはないが、特殊関数に興味がある人。
  \item 特殊関数を勉強したことはあるが、体系的に理解しきれていなかったりどこか釈然としない箇所を抱えている人。
  \item 専門分野を勉強する上で特殊関数が必要となり、そのレファレンスとして利用したい人。
\end{itemize}\vspace{12pt}
ただし、次に示すような人は読者として想定しておりません。 \vspace{3pt}
\begin{itemize}
  \item 特殊関数そのものや純粋数学を専門としており、数学的に厳密な理論を追究する人。
  \item 後述する「何が載っていないか」に示す項目を勉強したい人。
  \item レポートや論文を執筆しており、その中で引用する文献を求めている人。
\end{itemize}\vspace{3pt}
本稿は特殊関数をあくまで道具として利用する立場で書いています。
そのため、いくつかの説明では数学的に厳密ではない箇所が点在していると思われます。
また、いうまでもなく本稿は由緒ある学術書とは程遠く、
ただの一大学生によるメモ書きに過ぎないので、
学生レポートや論文の中に引用する文献としては不適切に思います。
本稿が被引用文献として利用される自信はまったくないのですが、
それでも読者の中に本稿に対して引用に近しいことをされたいという方がもしいらっしゃるのであれば、
後述する「参考書」で挙げた本をいくつか引っ張り出してそれを載せておけば
少なくとも形にはなると思います(\verb|^^;|)。



\section*{それぞれの項目について}
特殊関数で掌握すべき項目の概観をいくつか説明しておきましょう。

\subsection*{母関数}
ある数列$\{ a_n \}$について、次の式により定義される関数を\textbf{母関数}と呼びます。
\begin{equation*}
  g(t) = \sum_{n=0}^\infty a_n t^n
\end{equation*}
すなわち、数列をある種の級数の展開係数にしたときの収束先となる関数が母関数です。
このようなものを考えると数列についてのさまざまな考察が可能となるため、
強力な道具として重宝されます
\footnote{
たとえば、結城浩『数学ガール』(SB Creative)にはフィボナッチ数列の一般項を母関数で導く手法が
紹介されています。
}
。

母関数は関数列$\{ f_n(x) \}$に対しても定義することができます。この際、
母関数は$t$と$x$の2変数関数となります。たとえば、
\begin{equation*}
  g(t, x) = \sum_{n=0}^\infty f_n(x) t^n
\end{equation*}
などと定義されます。場合によっては$n!$を取っ払って
\begin{equation*}
  g(t, x) = \sum_{n=0}^\infty \frac{t^n}{n!} f_n(x)
\end{equation*}
などと定義されることもあります。
本稿では整数次の特殊関数を扱うときは母関数を定義として採用しています。
これからさまざまな公式が導かれるさまを眺めてみると、母関数がいかに強力であるかが理解できると思います。

\subsection*{Rodriguesの公式}
\textbf{Rodriguesの公式}とは$n$次の特殊関数を、ある関数の$n$階微分で表現した公式です。
一番有名なものはLegendre多項式についての次の公式でしょう。
\begin{equation*}
  P_n (x) = \frac{1}{2^n n!} \dx{n} (x^2 - 1)^n
\end{equation*}
Rodriguesの公式は小さい$n$に対する関数の具体的表現を求めやすいことや、
初等的な関数で表現した場合の大まかな形が理解しやすいなどの利点があります。
しかし$n$が大きくなると何度も微分を行わねばならず、あまり実用に向かなくなります。

\subsection*{一般項}
一般項とは特殊関数の次数を与えると、その具体的な級数展開式を得ることができる公式です。
一般項は整数次に限らず、一般次数の特殊関数についても考えることができます。
たとえば一般次数のBessel関数の一般項は次のように与えられます。
\begin{equation*}
  J_\nu(x) = \sum_{i=0}^\infty \frac{(-1)^i}{i!\ \Gamma(\nu +i+1)} \( \frac{x}{2} \)^{\nu+2i}
\end{equation*}
数列の問題では一般項を求めることが主題となる場合が多かったと思いますが、
ここで示したように特殊関数では多くの場合一般項が複雑になります。
そのため、関数の具体形を得る際に利用することはほとんどありません。
それでも一般項を掌握すべき理由は、
特殊関数の$x\sim 0$における振る舞いを理解しやすい形になっているからです。
たとえばBessel関数の場合、$x\sim 0$においては
上の式で$i=1$以降の項を無視することで、
次のように$\nu$次関数として振舞うであろうことが容易に分かります。
\begin{equation*}
  J_\nu (x) \sim \frac{1}{\Gamma(\nu+1)} \( \frac{x}{2} \)^\nu \qquad (x\sim 0)
\end{equation*}

\subsection*{漸化式}
数列の場合と同様に、特殊関数についても\textbf{漸化式}を考えることができます。
本稿で扱う特殊関数の漸化式はすべて隣接する3項の間の漸化式になっているので、
上で挙げたRodriguesの公式や一般項により最初の2項を求めておくと、
残りの項はすべて帰納的に導くことができます。
Rodriguesの公式や一般項と比べて少ない計算量で大きな次数に対する特殊関数の具体形を得ることが出来るので、
漸化式は実際の数値計算で最もよく用いられます。

本文中では漸化式のほかに\textbf{微分漸化式}と呼ばれるものを書いています。
これは一般的に使われている用語ではなく、
論理の道筋が明確になるように公式集にまとめて書くため、
その便宜をはかってその名前を付しています。
本文で行いますが、本稿では「漸化式」は母関数$g(t, x)$を$t$で微分することで得られる式、
「微分漸化式」は$x$で微分することで直ちに得られる式、という意味で用いています。

公式集ではさらに\textbf{昇降演算子}を記しています。
$n$次の関数$f_n(x)$から$n-1$次の関数$f_{n-1} (x)$を得る演算子は\textbf{下降演算子}、
逆に$n+1$次の関数$f_{n+1} (x)$を得る演算子は\textbf{上昇演算子}と呼ばれ、
ふたつを総称して昇降演算子と呼ばれます。
昇降演算子は2項間の漸化式となっていますが、そのかわりに微分作業を伴うという特色があります。
また、昇降演算子を導入することで次の微分方程式が自然に導かれます。


\subsection*{微分方程式}
実際の物理学や工学などの現場では微分方程式を解く問題に帰着することが多く、
その解に関心が寄せられます。
そのため普通の文脈では解くべき微分方程式を級数解法やFrobeniusの方法などを用いて解くことになります。
しかし本稿では多くの特殊関数の入門書がそうであるように、
母関数によって特殊関数を定義すると、その特殊関数が満たすべき方程式はこうですねという論理で
微分方程式を導きます。
実際の問題では、「現れた微分方程式はこの特殊関数の微分方程式と同じ形をしている」と見たとき、
方程式の解はその特殊関数になるとしてよいことになります。
微分方程式は昇降演算子を1回ずつ用いることで得ることができます。


\subsection*{直交性}
関数列$\{ f_n(x) \}$に属する
ふたつの関数$f_m(x), f_n(x)$が区間$[\alpha, \beta]$において直交するとは、$c_n$を$n$により定まる定数として
\begin{equation*}
  \int_\alpha^\beta f_m(x) f_n(x) dx = c_n \delta_{mn}
\end{equation*}
が成立することをいうのでした
\footnote{
この定積分は区間$[\alpha, \beta]$における関数$f_m(x), \ f_n(x)$の\textbf{内積}と呼ばれます。
}
。
関数列が直交性を持つとき、任意の関数はその関数列により級数展開できます
\footnote{
正確には直交する関数列が\textbf{完全性}を持っていることも示さなければなりません。
}
。
最も有名な例は三角関数や複素指数関数により展開するFourier級数です。


本稿で扱う特殊関数もすべて直交性を持っています
\footnote{
特殊関数によっては次のように直交性の公式に関数$w(x)$が余計についているものがあります。
\begin{equation*}
  \int_\alpha^\beta w(x) f_m(x) f_n(x) dx = c_n \delta_{mn}
\end{equation*}
$w(x)$は\textbf{重み関数}と呼ばれます。
このような性質を持つ場合も、関数に直交性があると解釈します。
}
。
そのためあらゆる関数を特殊関数により展開することが可能となります。
また、直交性の公式に現れる定数$c_n$は量子力学において規格化された波動関数
\footnote{
定義域において自身との内積が1となるように定めた波動関数のこと。
量子力学における確率解釈からの要請です。
}
を求める際に必要になります。




\section*{何が載っていないか}
以下の項目は、筆者の時間と気力と能力の関係でまだ本稿に記しきれていません。
要望があったり私が必要性を感じてきたときに加筆しようと思っています。

\begin{itemize}
  \item \textbf{積分表示}\\
  後述の一般次数の特殊関数を導入したり、漸近形を証明する際に重要な役割を占める公式です。\vspace{4pt}

  \item \textbf{一般次数の特殊関数}\\
  Bessel関数を除いて、本稿では多くの物理系で重要視される整数次の特殊関数しか述べられていません。
  正確に問題を理解するには、前述の積分表示により一般次数の特殊関数を導入して
  議論する必要があります。\vspace{4pt}

  \item\textbf{漸近形の証明}\\
  漸近形とは特殊関数の無限遠における振る舞いを表す公式です。
  $x\sim 0$における振る舞いは一般項から理解できますが、
  漸近形は前述の積分表示に対して鞍点法と呼ばれる手法を用いて証明することになります。\vspace{4pt}

  \item \textbf{微分方程式の級数解法}\\
  本来はこれもやるべきです。\vspace{4pt}

  \item \textbf{第2種の特殊関数}\\
  本稿で扱う特殊関数の微分方程式はすべて2階なので、2つの線形独立解をもつはずです。
  しかしBesselの微分方程式を除き、本稿で扱う特殊関数の微分方程式の解について、
  片方(第2種~関数などと呼ばれる)は  考えている領域で発散するなど、
  物理的に不適切なことが知られています。
  このことを理解するためには微分方程式の級数解法を行うか、
  先述の積分表示や後述する超幾何関数表示を用いて第2種の特殊関数を導入して
  議論しなければなりません。\vspace{4pt}

  \item \textbf{ロンスキアン}\\
  微分方程式の解の線形独立性を調べる際に必要となる概念です。\vspace{4pt}

  \item \textbf{超幾何関数表示}\\
  特殊関数を統一的に理解するときに重要な表示です。
  超幾何関数と呼ばれるものを用いると、  一般次数の場合や第2種を含めて、
  本稿で扱う特殊関数がすべて級数で表現できることを明確に示すことができます。\vspace{4pt}

  \item \textbf{変形球Bessel関数}
  \begin{table}[H]
    \centering \small
    \caption{偏微分方程式の解の$r$依存性として現れる関数}
    \begin{tabular}{c|cc}\Hline
      				& 円筒座標$(r, \theta, z)$ &  球座標$(r, \theta, \varphi)$ \\\hline
      波動方程式 	& Bessel関数 		& 球Bessel関数 \\
      拡散方程式	& 変形Bessel関数	& 変形球Bessel関数 \\\hline
    \end{tabular}
    \label{tab:i-k}
  \end{table}
  第4章で扱うBessel関数は、円対称、あるいは球対称な系の物理を考える際にしばしば現れる特殊関数です。
  具体的には波動方程式
  \footnote{
    ここでは一般的に知られる波動方程式$\(\Delta - \frac{1}{c^2} \frac{\6^2}{\6 t^2}\) \psi = 0$と
    Helmholtz方程式$\(\Delta + k^2\) \psi = 0$を総称して波動方程式と呼んでいます。
  }
  と拡散方程式
  \footnote{
  拡散方程式: 
    $\(\Delta - \frac{1}{\kappa} \frac{\6}{\6 t}\) \psi = 0$
  }
  を円筒座標$(r, \theta, z)$あるいは球座標$(r, \theta, \varphi)$で解く場合、
  解の$r$依存性として現れる関数は表\ref{tab:i-k}のようになります。

  本文中で述べるように、変形Bessel関数は純虚数変数についてのBessel関数、
  球Bessel関数はBessel関数の特殊な場合という関係があるので、
  これらに対する公式はBessel関数に対する公式から直ちに導くことができます。
  同様に変形球Bessel関数も純虚数変数についての球Bessel関数、
  あるいは変形Bessel関数の特殊な場合という関係にあるので、
  その公式も直ちに得られるわけです。
  しかし、多くの教科書では変形球Bessel関数が書かれていないこと、
  その代わりに球Bessel関数の引数を純虚数とした表記がなされることなどの観点から、
  本稿にこの関数を載せないことにしました。\vspace{4pt}

  \item \textbf{その他(参考)と付した公式の証明}\\
\end{itemize}

\section*{参考書}
本稿を執筆する際に参考にした教科書等を以下にあげておきます。
読んでいてよく分からない箇所があったり、さらに深く学びたいときに手に取ってみてみるとよいかもしれません。\vspace{6pt}
\begin{enumerate}
  \item \textbf{小野寺嘉孝『物理のための応用数学』(裳華房) }
  \item \textbf{小野寺嘉孝『基礎演習シリーズ 物理のための応用数学』(裳華房)} \\
  特殊関数の公式やその証明についてここまで略さず簡潔にまとめられた教科書や演習書は他にないと思います。
  本稿の証明の多くはこれら2冊を参照しています。\vspace{6pt}

  \item \textbf{後藤憲一 他『詳解 物理応用 数学演習』(共立)} \\
  ご存知、共立の演習書です。物理で必要な数学の知識がほぼすべてまとまっている数少ない本の1つと思います。
  しかし特殊関数については1つの章にまとめられておらず、
  Laguerre陪多項式の定義も章によって変わっていたりするので
  まだまだ問題はあると思いますが、持っていると心強いです。\vspace{6pt}

  \item \textbf{アルフケン, ウェーバー『基礎物理数学 vol.2 関数論と微分方程式』(講談社)}
  \item \textbf{アルフケン, ウェーバー『基礎物理数学 vol.3 特殊関数』(講談社)} \\
  物理数学の定番といわれるアルフケン。かなり分かりやすく具体例も豊富なため
  これを修得しておくと特殊関数で困ることはほとんど無くなると思います。
  ただ、訳書には章末問題の解答が付せられていないので注意。\vspace{6pt}

  \item \textbf{馬場敬之『スバラシク実力がつくと評判の偏微分方程式 キャンパス・ゼミ 改訂2』(マセマ)}\\
  偏微分方程式を道具として用いる立場の本として具体的で分かりやすい説明がなされたものは
  これしかないと思います。\vspace{6pt}

  \item \textbf{篠崎寿夫, 若林敏雄, 木村正雄 『工学者のための偏微分方程式とグリーン関数』(現代工学社)}\\
  基本的な偏微分方程式の解法がひととおりまとめられている本。
  特殊関数についても簡潔に触れられています。\vspace{6pt}

  \item \textbf{砂川重信『理論電磁気学』(紀伊国屋書店)}\\
  電磁気学の標準的な教科書。
  説明が丁寧で、電磁気学に現れる特殊関数についても学ぶことができます。\vspace{6pt}

  \item \textbf{岡崎誠, 藤原毅夫『演習 量子力学[新訂版]』(サイエンス社)}\\
  サイエンス社の黄色い本のシリーズの中で最も出来が良いといわれる演習書。
  量子力学の重要な問題がよくまとまっており、
  軽くめくるだけで量子力学の全体像をつかむことができると思います。\vspace{6pt}
% \CID{8362}
  \item \textbf{森口繁一, 宇田川銈久, 一松信『岩波 数学公式 III 特殊関数』(岩波書店)} \\
  有名な数学公式集。
  公式集なので証明は載ってないですが、知識を整理したりレファレンスとして利用したりするのに優れています。
  
  \vspace{12pt}
  インターネット上の有用な情報源としては次を挙げておきます。\vspace{6pt}

  \item \textbf{''Scientific Doggie 数理の楽しみ''} $<$\url{http://www.wannyan.net/scidog/}$>$\\
  しばわんころさんによるまとめノートが掲載されているサイト。
  おそらくインターネット上で閲覧できる特殊関数について日本語で書かれた資料の中で最も詳しいと思います。
  二重和の取りかえ方の説明などは大変参考になりました。\vspace{6pt}

  \item \textbf{''特殊関数とその応用について''} 
		$<$\url{http://www.sci.hyogo-u.ac.jp/maths/master/h11/kunimasa.pdf}$>$\\
  兵庫教育大学の学生による卒業論文と思われる資料。
  本稿に含まれるLegendre関数やBessel関数などについてまとめられています。\vspace{6pt}

  \item \textbf{''倭算数理研究所''} $<$\url{https://wasan.hatenablog.com/}$>$\\
  倭マンさんのブログ。特殊関数の記事では岩波数学公式の証明がいくつか書かれています。\vspace{6pt}

  \item \textbf{''特殊関数 グラフィックスライブラリー''} $<$\url{http://math-functions-1.watson.jp}$>$\\
  Souichiro Ikebeさんによる、特殊関数のグラフが数多く掲載されているサイト。
  次数によって関数のグラフがどのように変化するのかについても人目で分かるようになっています。\vspace{6pt}

  \item \textbf{''SymPy 1.3 documentation''} $<$\url{https://docs.sympy.org/latest/index.html}$>$\\
  本稿のグラフはSymPyを用いて描かれていますが、その公式レファレンスサイトです。\vspace{6pt}

  \item \textbf{''Python Kugelfl\"{a}chenfunktionen spherical harmonics''}\\ 
$<$\url{http://www.magben.de/?h1=mathematik_fuer_ingenieure_mit_python&h2=kugelflaechenfunktionen}$>$\\
  球面調和関数や軌道の可視化の際にここを参照しました。

\end{enumerate}

\section*{はじめにのおわりに}

本稿は専門家でもなんでもない大学生のメモ書きということもあり、
不適切な記述や誤りなどがあると思われます。
読者の皆さんから本稿の中の誤りのご指摘やご意見等をいただければうれしく思います。

特殊関数はそれ自体の理論はもちろんのこと、実際の物理や工学などへの応用のされ方も
非常に興味深い物理数学の一分野です。
本稿により読者が特殊関数の学習を少しでも楽にできたのであれば、
筆者としてこれ以上の幸福はありません。




\end{document}